\providecommand{\main}{../../main}
\documentclass[../../main.tex]{subfiles}


\begin{document}

\chapter{Conclusion and future developments}
\label{sec:Conclusion}
In this thesis it was shown the implementation of new algorithm conceived for the Level-1 Trigger Phase-2 upgrade, namely the neural network based conditions. Firstly, a detailed study of the implementation and relative integration of such novel algorithm has been carried out in order to correctly interface with the Phase-2 global trigger framework. Secondly the last component of the global trigger chain, the final OR board, has been presented with a particular focus on the firmware development done to cope with the new physics requirements.  

In section \ref{sec:Gt-algo-board} an overview of the Global Trigger algorithm board is given highlighting its intrinsic complexity due to the requested scalability.  
In section \ref{sec:P2GT_NN} the target neural network model is explained in depth, afterwords the optimization steps were described in detail to reach the resource reduction for the hardware deployment. 

Firstly the model hyper-parameter has been quantized, this was done thanks to the \textit{QKeras} and \texit{hls4ml} packages. Secondly, the redundant connections between nodes have been pruned to lightening the target model. Finally its translation into hardware language took place. During each step multiple tests were carried out to spot any potential performance degradation.

Hardware interface between the GT framework and the neural network model has been developed where multiple modules have to be instantiated to compute the same pre-processing done in software.

The complete module is then tested within the GT framework, using time multiplexed data. Little to no performance degradation was seen with respect to the baseline model entirely computed on software, as it is shown in section \ref{sec:P2GT_Res}. The complete integration was proven successful with many different neural network architectures. The latency of 30ns was kept well under the requirement set by the global trigger,  while its resource footprint was kept at minimum level thanks to the optimizations introduced.

Two different target frequencies were considered: 480MHz (same as the global trigger logic) and 240MHz. Advantages and drawbacks of each option were studied and 240MHz was chosen as target frequency due to small increased in logic complexity but lower flip flop usage and simpler routability without sacrificing the latency.


The next important achievement would be the integration of hls4ml on the CMS offline software (CMSSW), in this way the neural network module can be developed and integrated within the CMS software stack without going through the technical steps of firmware development. An other important milestone will be the emulation of such block in CMSSW allowing rapid test and fast development of new algorithms. 


The final OR board firmware was developed and then tested, the design took as starting point the $\mu$GT firmware running during the Run 3. It was optimized and improved to meet the new requirement of the Phase-2 L1T upgrade. The main blocks have been lightened were possible to meet the increased number of running algorithms and the increased data throughput.  

New tests were developed to evaluate the qualities of the produced design in multiple hardware configurations, from standalone to multi-board that mimic the final GT hardware chain. A small trigger menu was introduced to test the measured rates against the expected ones. The board positively passed every tests with various configurations and data inputs.

Next step before the installation of the mentioned board is the implementation of the missing blocks described in the Phase-2 TDR\cite{L1T-2up}, namely the external conditions trigger, the connection to the DAQ boards and the output for the HLT subsystem.


    

\end{document}